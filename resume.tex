%%%%%%%%%%%%%%%%%%%%%%%%%%%%%%%%%%%%%%%%%
% Important note:
% This template requires the resume.cls file to be in the same directory as the
% .tex file. The resume.cls file provides the resume style used for structuring the
% document.
%
%%%%%%%%%%%%%%%%%%%%%%%%%%%%%%%%%%%%%%%%%

% +--------------------------------------------+
% | PACKAGES AND OTHER DOCUMENT CONFIGURATIONS |
% +--------------------------------------------+

\documentclass{resume} % Use the custom resume.cls style

\usepackage[left=0.75in,top=0.6in,right=0.75in,bottom=0.6in]{geometry}

\newcommand{\tab}[1]{\hspace{.2667\textwidth}\rlap{#1}}
\newcommand{\itab}[1]{\hspace{0em}\rlap{#1}}

\name{Nicholas W. Breitling}
\address{744 Applewood Cir., Victoria, MN 55386}
\address{(+1) 952-905-4220 \\ breitnw@u.northwestern.edu}

\begin{document}

% +-------------------+
% | EDUCATION SECTION |
% +-------------------+

\begin{rSection}{Education}
  \begin{rItem}{Northwestern University}{Evanston, IL}{Sep. 2024}{Present}
    \item GPA: 4.0/4.0
    \item Coursework: Programming Languages, Data Structures and Algorithms, Computer Systems
  \end{rItem}

  \begin{rItem}{Northeastern University}{Boston, MA}{Sep. 2024}{Present}
    \item GPA: 4.0/4.0
    \item Coursework: Accel. Fundamentals of Computer Science 1 and 2, Accel. Discrete Strctures, Intensive Math Reasoning, Logic \& Computation
  \end{rItem}

  \begin{rItem}{Minnetonka High School}{Minnetonka, MN}{Sep. 2019}{Jun. 2024}
    \item GPA: 4.615 (W), 4.0 (UW); ACT: 36
    \item Coursework: AP CS A, AP CS Principles, Calculus 1-4, Linear Algebra
    \item Awards: National Merit Scholarship Award, Dartmouth Alumni Club Book Award, AP Scholar with Distinction, Summa Cum Laude
  \end{rItem}
\end{rSection}

% +-----------------------+
% | RESEARCH AND PROJECTS |
% +-----------------------+

\begin{rSection}{Research \& Projects}
  % PRL-PRG research
  \begin{rItem}{Research Assistant}{PRL-PRG}{Jan. 2024}{Sep. 2024}
    \item Collaborated on the development of a new infrastructure for just-in-time compilation of the R programming language
    \item Aided in transition from C++ to a Java compile server, focusing specifically on RDS serialization
    \item After completion of RDS writer, integrated serialization system as a communication protocol between C++ frontend and Java backend, enabling comprehensive integration testing via package compilation

    % ADDITIONAL NOTES
    % RDS writer PR: https://github.com/PRL-PRG/r-compile-server/pull/16
    % Integration test branch: https://github.com/PRL-PRG/r-compile-server/tree/backend-rds
    %
    % 1. Implemented serialization in the GNU R bytecode format, entailing a mapping from our bytecode to the GNU R standard, a compliant instruction serialization function, and cycle-aware constant pool serialization.
    % 2. Added full support for R s-expressions, including closures, promises, environments, vectors, lists, symbols, and builtins.
    % 3. Developed a comprehensive logging suite for debugging bytecode serialization and deserialization.
    % 4. Created a roundtrip test utility which deserializes each function in the R standard library and serializes it back to RDS. Validates the RDS reader and writer by checking that the writer's output for each function matches the original RDS input.
    % 5. Utilized RDS reader and writer to create integration tests for server-side bytecode compilation. For each, the C++ client sends a compilation request to the Java server, including a payload in the form of a standard R package (NAMESPACESXP). The server uses the RDS reader to parse the package, compiles it to bytecode, serializes that bytecode in the GNUR format using the RDS writer, and sends it back to the client. The client then checks whether the bytecode received from the server matches the output of the standard, client-side bytecode compiler. If they do not match, a diff is printed.
  \end{rItem}

  % Minnetonka Research
  \begin{rItem}{Independent Researcher}{Minnetonka Research}{Sep. 2022}{May 2023}
    \item Project title: {\em Improving sphere blending performance for fluid simulation applications using ray-marched rendering}
    \item Worked with Vulkan to develop ray-marched and mesh rendering engines, utilizing a smooth-min distance field and a marching-cubes algorithm, respectively, for sphere blending.
    \item Compared performance metrics for both algorithms, highlighting efficiency of ray-marching
    \item Twin Cities Regional Science Fair: awarded blue ribbon (first place in Systems Software category), purple ribbon (advancement to State), and Stockholm Junior Water Prize
  \end{rItem}

  % Other projects
  {\bf Other projects} {\em (full list at github.com/breitnw)}
  \begin{itemize}
    \item \parbox{3cm}{\em mndco11age.xyz:} Portfolio website and webserver; developed with Rust and OpenSSL
    \item \parbox{3cm}{\em rhyolite:} Apple Music miniplayer and queue manager, developed with Rust, C, and SDL2
    \item \parbox{3cm}{\em micromusic:} Vulkan-based mesh rendering engine, developed with Rust and GLSL
  \end{itemize}

\end{rSection}

% +--------------------+
% | EXPERIENCE SECTION |
% +--------------------+

\begin{rSection}{Experience}
  % Humanity Alliance volunteering
  \begin{rItem}{Project Lead}{The Humanity Alliance}{May 2021}{May 2023}
    \item Service internship with the Humanity Alliance, non-profit delivering 9,000 meals/wk to food-insecure families
    \item Developed and programmed a full-stack, user-friendly dashboard to streamline integration of meal requests and delivery data with an interactive map, greatly reducing manual entry time by automating route calculation and assignment
    \item Maintained communication with organization leadership to address needs for  administration, user permissions and security
  \end{rItem}

  % CodeNinjas
  \begin{rItem}{Camp Counselor (``Sensei'')}{Code Ninjas}{Summers 2021}{2023}
    \item Lead counselor for weekly camps throughout summer. Provided one-on-one and presentational instruction to guide campers through curriculum and difficult concepts related to programming and application development
    \item Planned and implemented supplemental lessons in Lua and Scratch programming, 3D modeling, music distribution, and more, fostering an engaging environment for advanced campers
  \end{rItem}
\end{rSection}

% +------------------+
% | EXTRACURRICULARS |
% +------------------+

\begin{rSection}{Extracurriculars}
  \begin{rItem}{Volunteer CS Educator}{Evanston-Skokie School District 65}{Jan. 2025}{Present}
    \item Teach weekly computer science classes to 5th graders at Oakton Elementary School as part of education research conducted by Northwestern’s TIDAL and tiilt labs
    \item Introduce students to concepts such as loops, variables, and debugging while fostering self-expression through use of TunePad, a Python-based music production tool
  \end{rItem}

  \begin{rItem}{eCVT Developer}{Northwestern Baja SAE}{Oct. 2024}{Present}
    \item Develop ESP32 microcontroller software for eCVT (electronic continuously-variable transmission)
    \item Configure and tune hall-sensor and linear encoder PID inputs, maintaining optimal output RPM
  \end{rItem}

  \begin{rItem}{Curriculum Developer}{StemOUT}{Jan. 2024}{Apr. 2024}
    \item Developed an educational curriculum for elementary (K-5) schoolers with the goal of ``teaching AI without computers'', including interactive lessons on history, functions, and ethics. Taught this curriculum and others at public libraries.
  \end{rItem}

  \begin{rItem}{Captain}{FIRST Robotics Team 3082}{Sept. 2019}{May 2023}
    \item Oversaw electronics and programming subteams; led the development of an OpenCV-based stereoscopic vision system, physically-modeled robot simulation, Swerve drivetrain, inverse-multiplexed button board, and other subsystems
    \item Won Innovation in Control award, progressed to FRC World Championship during 2023 season
  \end{rItem}

  {\bf Other:} Scouts BSA, Eagle Rank; Tonka Hacks Hackathon, 1st place; NHS; Symphonic Band

\end{rSection}

% +----------------+
% | SKILLS SECTION |
% +----------------+

\begin{rSection}{Skills}
  {\bf Programming Languages and Frameworks }
  \begin{itemize}
    \item \parbox{3cm}{\em Functional:} Racket, Haskell
    \item \parbox{3cm}{\em Systems:} Rust, C, x86 Assembly, C++
    \item \parbox{3cm}{\em Web/App Dev:} Java, JavaScript, Python, Flask, Jinja, JQuery
    \item \parbox{3cm}{\em Graphics:} Vulkan, OpenGL
    \item \parbox{3cm}{\em Config/VCS:} Nix, Elisp, Make, Git
  \end{itemize}
\end{rSection}

\end{document}
